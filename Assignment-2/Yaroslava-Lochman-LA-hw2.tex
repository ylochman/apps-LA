
\documentclass[12pt,a4]{article}
%%%%%%%%%%%%%%%%%%%%%%%%%%%%%%%%%%%%%%%%%%%%%%%%%%%%%%%%%%%%%%%%%%%%%%
\usepackage{amsmath,amsthm}
\usepackage{amsfonts}
\usepackage{amssymb}
\usepackage[normalem]{ulem}
\usepackage{enumerate}
\usepackage{graphicx}%
\usepackage{datetime,verbatim}
\usepackage{xcolor}

%%%%%%%%%%%%%%%%%%%%%    page setup   %%%%%%%%%%%%%%%%%%%%%%%%%%%%%%% 

\textheight=235truemm \textwidth=175truemm \hoffset=-15truemm
\voffset=-15truemm

%%%%%%%%%%%%%%%%%%%%%%%%%%%%%%%%%%%%%%%%%%%%%%%%%%%%%%%%%%%%%%%%%%%%%
\usepackage{fancyhdr}
\pagestyle{fancy}
\lhead{{\sf\scriptsize R.H.}}
\rhead{{\sf\scriptsize Autumn~2018}}
\chead{{\sf\scriptsize Linear Algebra: Homework~2 @  DS UCU%~\Versia
}}
\lfoot{}
\rfoot{}
\cfoot{\rm\thepage}
%\pagestyle{myheadings}
%%\markright{}
%\markboth{}{}
%%%%%%%%%%%%%%%%%%%%%%%%%%%%%%%%%%%%%%%%%%%%%%%%%%%%%%%%%%%%%%%%%%%%%


%%%%%%%%%%%%%%%%%%%%%%%%%%%%%%%%%%%%%%%%%%%%%%%%%%%%%%%%%%%%%%%%%%%%%


%%%%%%%%%%%%%%%   matrix extension  %%%%%%%%
\makeatletter
\renewcommand*\env@matrix[1][*\c@MaxMatrixCols c]{%
	\hskip -\arraycolsep
	\let\@ifnextchar\new@ifnextchar
	\array{#1}}
\makeatother
%%%%%%%%%%%%%%%%%%%%%%%%%%%%%%%%%%%%%%%%%%%%

\newenvironment{proofNoQED}[1]{\smallskip\noindent{\it Proof #1.}\ \rm}
{\hfill \smallskip}
\newcommand{\ProofNoQED}[1]{\smallskip\noindent{\it Proof} #1\ \hfill\smallskip}
%\renewcommand{\qedsymbol}{\text{$\square$}}

\newtheorem{problem}{Problem}
\newtheorem{solution}{Solution to the problem}

%%%%%%%%%%%%%%%%%%%%%%%%%%%%%%%%%%%%%%%%%%%%%%%%%%%%%%%%%%%%%%%%%%%%%

%%%%%%%%%%%%%%%%%%%%%%%%%%%%%   Definitions       %%%%%%%


\newcommand\rank{\operatorname{rank}}
\newcommand{\trace}{\operatorname{tr}}
\newcommand\grad{\operatorname{grad}}
\newcommand\ls{\operatorname{ls}}

\newcommand{\ov}{\overline}
\newcommand{\wt}{\widetilde}

\newcommand{\bN}{{\mathbb N}}
\newcommand{\bR}{{\mathbb R}}
\newcommand{\bZ}{{\mathbb Z}}

\newcommand{\ba}{{\mathbf a}}
\newcommand{\bb}{{\mathbf b}}
\newcommand{\be}{{\mathbf e}}
\newcommand{\bn}{{\mathbf n}}
\newcommand{\bff}{{\mathbf f}}

\newcommand{\bu}{{\mathbf u}}
\newcommand{\bv}{{\mathbf v}}
\newcommand{\bp}{{\mathbf p}}
\newcommand{\bq}{{\mathbf q}}
\newcommand{\br}{{\mathbf r}}
\newcommand{\bx}{{\mathbf x}}
\newcommand{\by}{{\mathbf y}}
\newcommand{\bw}{{\mathbf w}}

\newcommand{\cB}{{\mathcal B}}
\newcommand{\cF}{{\mathcal F}}
\newcommand{\cG}{{\mathcal G}}
\newcommand{\cN}{{\mathcal N}}
\newcommand{\cP}{{\mathcal P}}
\newcommand{\cT}{{\mathcal T}}

\newcommand{\sprod}[2]{\left \langle #1, #2 \right \rangle}
\newcommand{\norm}[1]{\left\lVert#1\right\rVert}
\newcommand{\abs}[1]{\left | #1 \right |}
\newcommand{\vect}[1]{\overrightarrow{#1}}

%%%%%%%%%%%%%%%%%%%%%%%%%%%%%%%%%%%%%%%%%%%%%%%%%%%%%%%%%%%%%%%%%%%%%


\begin{document}

\begin{center}
  \Large\bf{Linear Algebra\\
    Home assignment 2: Orthogonality}
\end{center}
Solutions are by Yaroslava Lochman.

%\vspace{.5cm}

\begin{problem}[Parallel and orthogonal planes; 2pt]\rm	Determine whether the given planes are:
		\begin{enumerate}[(a)]
		\item parallel: \begin{enumerate}[(i)]
			\item $4x-y+2z=5$ and $7x-3y+4z=8$;
			\item $x-4y-3z-2=0$ and $3x-12y-9z-7=0$.
		\end{enumerate}
 		\item perpendicular:
		\begin{enumerate}[(i)]
			\item $3x-y+z=0$ and $x+2z=-1$;
			\item $x-2y+3z=4$ and $-2x+5y+4z=-1$.
		\end{enumerate}
	\end{enumerate}
\end{problem}

\begin{solution}[]\rm 
\[
Ax + By + Cz = D
~\Rightarrow~
n = (A,B,C)^\top \text{ -- the normal vector of the plane}
\]
The planes are parallel $~\Leftrightarrow~$ the normal vectors are parallel $~\Leftrightarrow~$ $\frac{A_1}{A_2}=\frac{B_1}{B_2}=\frac{C_1}{C_2}$.\\
The planes are perpendicular $~\Leftrightarrow~$ the normal vectors are perpendicular $~\Leftrightarrow~$ $\sprod{n_1}{n_2} = 0$$~\Leftrightarrow~$ $A_1A_2+B_1B_2+C_1C_2 = 0$
\begin{enumerate}[(a)]
\item (parallel)
\begin{enumerate}[(i)]
\item 
\[
4x - y + 2z = 5
~\Rightarrow~
n_1 = (4, -1, 2)^\top
\]
\[
7x - 3y + 4z = 8
~\Rightarrow~
n_2 = (7, -3, 4)^\top
\]
\[
\frac{4}{7} \neq \frac{-1}{-3} \neq \frac{2}{4}
\]
Hence the planes are not parallel.
\item
\[
x - 4y - 3z - 2 = 0
~\Rightarrow~
n_1 = (1, -4, -3)^\top
\]
\[
3x - 12y - 9z - 7 = 0
~\Rightarrow~
n_2 = (3, -12, -9)^\top
\]
\[
\frac{3}{1} = \frac{-12}{-4} = \frac{-9}{-3} = 3
\]
Hence the planes are parallel.
\end{enumerate}
\item (perpendicular)
\begin{enumerate}[(i)]
\item For $3x - y + z = 0$ and $x + 2z = -1$ we have
\[
\sprod{n_1}{n_2} = 3 + 0 + 2 = 5 \neq 0
\]
Hence the planes are not perpendicular.
\item For $x - 2y + 3z = 4$ and $-2x + 5y + 4z = -1$ we have
\[
\sprod{n_1}{n_2} = -2 - 10 + 12 = 0
\]
Hence the planes are perpendicular.\\
\end{enumerate}
\end{enumerate}
\end{solution}
	
\begin{problem}[Orthogonal complement; 3pt]\rm
	\begin{enumerate}[(a)]
		\item Let $W$ be the plane in $\bR^3$ given by the equation $x-2y-3z=0$. Find parametric equations for $W^\perp$.
		\item Let $W$ be the line in $\bR^3$ with parametric equations $x=2t$, $y=-5t$, $z=4t$. Find an equation for $W^\perp$.
		\item Let $W$ be the intersection of the two planes $x+y+z=0$ and $x-y+z=0$ in $\bR^3$. Find an equation for $W^\perp$.
	\end{enumerate}
\end{problem}


\begin{solution}[]\rm .
\begin{enumerate}[(a)]
\item
\[
% W = \{(x,y,z)^\top | x-2y-3z=0 \} \Leftrightarrow \{(x,y,z)^\top | \sprod{(x,y,z)^\top}{(1,-2,-3)^\top}=0 \}
W = \{ \begin{pmatrix} x & y & z \end{pmatrix}^\top \in \mathbb{R}^3 ~\vline~ x-2y-3z=0 \}
\qquad
\bn = \begin{pmatrix} 1 & -2 & -3 \end{pmatrix}^\top \text{ -- normal vector of the plane}
\]
\[
\text{So ~} \bx \in W ~\Rightarrow~ \sprod{\bx}{\bn}=0
\]
\[
\by \in W^\perp \Rightarrow \by ~\bot ~W ~\Rightarrow~ \sprod{\by}{\bx}=0 ~\forall \bx \in W ~\Rightarrow~ \by \parallel \bn
\]
So:
\[
W^\perp = \left\{\begin{pmatrix} x & y & z \end{pmatrix}^\top \in \mathbb{R}^3
~\vline~
\begin{matrix}
x=t \quad\\
y=-2t \\
z=-3t \\
\end{matrix}
\quad t \in \mathbb{R}\right\} = \ls\{\bn\}
\]
The parametric equations for $W^\perp$ are $x=t$, $y=-2t$, $z=-3t$.
\item
\[
W = \left\{\begin{pmatrix} x & y & z \end{pmatrix}^\top \in \mathbb{R}^3
~\vline~
\begin{matrix}
x=2t ~\\
y=-5t \\
z=4t ~\\
\end{matrix}
\quad t \in \mathbb{R}\right\}
\qquad
\bv = \begin{pmatrix} 2 \\ -5 \\ 4 \end{pmatrix} \text{ -- direction vector of the line}
\]
\[
\by \in W^\perp ~\Rightarrow~ \sprod{\by}{\bx}=0 ~\forall \bx \in W ~\Rightarrow~ \by \bot \bv \Rightarrow W^\perp \text{ is a plane with the normal vector } \bv
\]
So:
\[
W^\perp = \{ \begin{pmatrix} x & y & z \end{pmatrix}^\top \in \mathbb{R}^3 ~\vline~ 2x-5y+4z=0 \}
\]
The equation for $W^\perp$ is $2x-5y+4z=0$.
\item
% Let $W$ be the intersection of the two planes $x+y+z=0$ and $x-y+z=0$ in $\bR^3$. Find an equation for $W^\perp$.
\[
W = \left\{\begin{pmatrix} x & y & z \end{pmatrix}^\top \in \mathbb{R}^3
~\vline~
% \left \{
\begin{matrix}
x+y+z=0 \\
x-y+z=0 \\
\end{matrix}
% \right.
\right\}
\]
$\bn_1 = \begin{pmatrix} 1 & 1 & 1 \end{pmatrix}^\top$ -- normal vector of the $1^{st}$ plane.\\
$\bn_2 = \begin{pmatrix} 1 & -1 & 1 \end{pmatrix}^\top$ -- normal vector of the $2^{nd}$ plane.
\[
W = \left\{\bx \in \mathbb{R}^3
~\vline~
\sprod{\bx}{\bn_1}=0 ~\wedge~
\sprod{\bx}{\bn_2}=0 
\right\} \text{ -- is a line}
\]
We have: $\bx ~\bot~ \bn_1 $, $ \bx ~\bot~ \bn_2$. Let $\by \in \ls\{\bn_1, \bn_2\} \Rightarrow \sprod{\by}{\bx}=\sprod{\alpha_1\bn_1 + \alpha_2\bn_2}{\bx} = \alpha_1 \cdot~ 0 + \alpha_2 \cdot~ 0 = 0$ 
$~\Rightarrow~ \bx ~\bot~ \ls\{\bn_1, \bn_2\}$.
Hence:
\[
W^\perp = \ls\{\bn_1, \bn_2\}
\]
To get an equation for the plane we need to compute its normal $\bn$ (which is concurrently a direction vector of the line $W$). We can do it using the cross product:
\[
\bn = \bn_1 \times \bn_2 = 
\begin{vmatrix}
e_1 & e_2 & e_3 \\
1 & 1 & 1 \\
1 & -1 & 1 \\
\end{vmatrix}
=
\begin{pmatrix}
2 \\
0 \\
-2 \\
\end{pmatrix}
\]
So the equation for $W^\perp$ is $2x -2z = 0$.\\
% and since 
% $ \by \in W^\perp ~\Rightarrow~ \sprod{\by}{\bx}=0 ~\forall \bx \in W$
\end{enumerate}
\end{solution}


\begin{problem}[Distance from a point; 4pt]\rm 
	\begin{enumerate}[(a)]		
		\item Find the distance from the point $P=(1,1,0)$ to the line $\frac{x-1}{2}=\frac{y-2}{1} = \frac{z+1}{2}$. 
		\item Let $\pi$ be a plane given by the equation $ax + by + cz +d = 0$ and $P(x_0,y_0,z_0)$ be a point outside it. Prove that the distance from $P$ to $\pi$ is given by the formula
		\[
			\frac{|ax_0 + by_0 + cz_0 + d|}{\sqrt{a^2 + b^2 + c^2}}.
		\] 
		\small{\textsf{Hint: if $Q$ is the point on $\pi$ realizing the distance, then $\overrightarrow{PQ}$ is collinear to $\mathbf{n}=(a,b,c)$ (why?). Take now any point $Q'$ on $\pi$ and find a projection of $\overrightarrow{PQ'}$ onto direction~$\mathbf{n}$}}
		\item Find the distance between the point~$P=(1,0,1)$ and the plane~$2x+2y-z=2$.
	\end{enumerate}
\end{problem}


\begin{solution}[]\rm .
\begin{enumerate}[(a)]		
\item
\[
P = (1, 1, 0)
\qquad
l: \frac{x-1}{2}=\frac{y-2}{1} = \frac{z+1}{2}
\qquad
\bv = \begin{pmatrix} 2 & 1 & 2 \\ \end{pmatrix}^\top
\text{ -- ~direction vector}
\]
Let
$O = (1, 2, -1)$.
% $O=\begin{pmatrix} 1 & 2 & -1\end{pmatrix}^\top$.
This point lies on the line. Then $\vect{OP} =  \begin{pmatrix} 0 & -1 & 1 \\ \end{pmatrix}^\top$. Let $P'$ be the point on the line realizing distance. Hence the distance can be found as:
\[
\rho = \abs{\vect{PP'}} = \sqrt{\abs{\vect{OP}}^2 - \abs{\vect{OP'}}^2}
\]
$\abs{\vect{PP'}}$ is the projection of $\vect{OP}$ on the line:
\[
\abs{\vect{OP'}}^2 = 
pr^2_\bv \vect{OP} = 
\frac{\sprod{\vect{OP}}{\bv}^2}{\norm{\bv}^2} =
\frac{(0-1+2)^2}{2^2 + 1^2 + 2^2} = \frac{1}{9}
\]
So:
\[
\abs{\vect{PP'}} = \sqrt{2 - \frac{1}{9}} = \frac{\sqrt{17}}{3}
\]
\item 
\[
P = (x_0, y_0, z_0)
\qquad
\pi:~ ax + by + cz + d = 0
\qquad
\bn = \begin{pmatrix} a & b & c\end{pmatrix}^\top
\text{ -- ~normal vector}
% P = \begin{pmatrix} x_0 & y_0 & z_0\end{pmatrix}^\top
\]
Let $Q$ be the point on $\pi$ realizing the distance. Since the distance is the shortest:
\[
\vect{PQ}~\bot~\vect{QQ'} \quad \forall Q' \in \pi \quad \text{or} \quad  \vect{PQ}~\bot~\pi 
\]
($Q \in \pi$ minimizes $\abs{PQ} \Leftrightarrow PQ~\bot~\pi$ -- had been concluded using the Pythagorean theorem).
Therefore $\vect{PQ}$ is collinear to the normal $\bn$.
So the distance $\rho=\abs{\vect{PQ}}$ is an absolute value (since the angle between vectors can be acute or obtuse) of the projection of $\vect{PQ'}$ onto $\bn$:
\[
Q' = \begin{pmatrix} x_q & y_q & z_q\end{pmatrix}^\top \in \pi
~\Rightarrow~
ax_q + by_q + cz_q + d = 0
\]
\[
\vect{PQ'} = \begin{pmatrix} x_q-x_0 & y_q-y_0 & z_q-z_0\end{pmatrix}^\top
\]
\[
\Rightarrow ~ \rho = \abs{pr_n \vect{PQ'}} =
\frac{\abs{\sprod{\vect{PQ'}}{n}}}{\norm{n}} =
\frac{\abs{\sprod{\begin{pmatrix} x_q-x_0 \\ y_q-y_0 \\ z_q-z_0\end{pmatrix}}{\begin{pmatrix} a \\ b \\ c\end{pmatrix}}}}{\sqrt{a^2+b^2+c^2}} =
\frac{\abs{- d - a x_0 - b y_0 - c z_0}}{\sqrt{a^2+b^2+c^2}}
\]
So:
\[
\rho = \frac{|ax_0 + by_0 + cz_0 + d|}{\sqrt{a^2 + b^2 + c^2}}.
\] 
\item
\[
P=(1,0,1)
\qquad
\pi: ~2x+2y-z-2=0
\]
\[
\rho = \frac{|ax_0 + by_0 + cz_0 + d|}{\sqrt{a^2 + b^2 + c^2}} = \frac{|2 +2 \cdot 0 -1 + -2|}{\sqrt{2^2 + 2^2 + (-1)^2}} = \frac{1}{\sqrt{9}} = \frac{1}{3}
\] 
\end{enumerate}
\end{solution}


\begin{problem}[Cross product; 4pt]\rm
\begin{enumerate}[(a)]
	\item For any two vectors $\bu  =(u_1,u_2,u_3)^\top$ and $\bv = (v_1,v_2,v_3)^\top$, their \textbf{\emph{vector product}}, or \textbf{\emph{cross product}} $\bu \times \bv$ is the vector $\bw = (w_1,w_2,w_3)^\top$ with entries
	\[
	w_1 = \begin{vmatrix}u_{2}&u_{3}\\v_{2}&v_{3}\end{vmatrix}, \qquad
	w_2 = - \begin{vmatrix}u_{1}&u_{3}\\v_{1}&v_{3}\end{vmatrix},\qquad
	w_3 = \begin{vmatrix}u_{1}&u_{2}\\v_{1}&v_{2}\end{vmatrix}.
	\]
	Prove that $\bw$ is orthogonal to both $\bu$ and $\bv$ in the sense that
	$\bw^\top \bu = \bw^\top \bv = 0$.
	
	{\small{\textsf{Hint: these products are cofactor expansions of some $3\times 3$ matrices}}}
	\item Assume that $\bu_1,\dots, \bu_{n-1}$ are linearly independent vectors in $\mathbb{R}^n$. Find a formula analogous to that in part (a) for a vector that is orthogonal to the subspace spanned by $\bu_1,\dots,\bu_{n-1}$.
	\end{enumerate}
\end{problem}

\begin{solution}[]\rm .
\begin{enumerate}[(a)]
\item 
\[
\bw = \bu \times \bv =
\begin{vmatrix}
e_1 & e_2 & e_3 \\
u_1 & u_2 & u_3 \\
v_1 & v_2 & v_3 \\
\end{vmatrix}
\]
\[
\bw^\top \bu = 
\begin{vmatrix}
u_1 & u_2 & u_3 \\
u_1 & u_2 & u_3 \\
v_1 & v_2 & v_3 \\
\end{vmatrix} = 0~\text{ since the $1^{st}$ and $2^{nd}$ rows are equal (therefore dependent)}
\]
The same with $\bv$:
\[
\bw^\top \bv = 
\begin{vmatrix}
v_1 & v_2 & v_3 \\
u_1 & u_2 & u_3 \\
v_1 & v_2 & v_3 \\
\end{vmatrix} = 0~\text{ since the $1^{st}$ and $3^{rd}$ rows are equal (therefore dependent)}
\]
\item Let $\bu_i = (u_i^1, \cdots, u_i^n)^\top$. Then the answer is: 
\[
\bw = 
\begin{vmatrix}
e_1 & \cdots & e_n \\
u_1^1 & \cdots & u_1^n \\
\vdots & & \vdots \\
u_{n-1}^1 & \cdots & u_{n-1}^n \\
\end{vmatrix}
\]
One can see that $\bw$ is orthogonal to all $\bu_i$:
\[
\bw^\top \bu_i = 
\begin{vmatrix}
u_i^1 & \cdots & u_i^n \\
u_1^1 & \cdots & u_1^n \\
\vdots & & \vdots \\
u_{n-1}^1 & \cdots & u_{n-1}^n \\
\end{vmatrix} = 0
\quad \forall i \in \ov{1,n} ~\text{ (because of the two equal rows)}
\]
Therefore it is orthogonal to $\ls\{\bu_1,\dots,\bu_{n-1}\}$:
\[
\forall \bx \in \ls\{\bu_1,\dots,\bu_{n-1}\} \quad \bw^\top\bx = \bw^\top \sum_{i=1}^{n-1}\alpha_i \bu_i = \sum_{i=1}^{n-1}\alpha_i \bw^\top \bu_i = 0
\]\\
\end{enumerate}
\end{solution}



\begin{problem}[Orthogonal matrices; 4pt]\rm
	\begin{enumerate}[(a)]
	\item 
	If $Q_1$ and $Q_2$ are orthogonal matrices, show that $Q_1^{-1}$ and $Q_1Q_2$ are orthogonal as well.
	\item
	Prove that an orthogonal matrix that is also upper-triangular must be diagonal.
	\end{enumerate}
\end{problem}

\begin{solution}[]\rm .
\begin{enumerate}[(a)]
\item 
$Q_1$ and $Q_2$ are orthogonal $\Leftrightarrow$ $Q_1^\top = Q_1^{-1}$, $Q_2^\top = Q_2^{-1}$. So:
\[
1.~(Q_1^{-1})^\top = (Q_1^\top)^\top  = Q_1 = (Q_1^{-1})^{-1}
\]
Hence $Q_1^{-1}$ is orthogonal.
\[
2.~(Q_1Q_2)^\top = Q_2^\top Q_1^\top = Q_2^{-1} Q_1^{-1} = (Q_2Q_1)^{-1}
\]
Hence $Q_1Q_2$ is orthogonal.
\item
Let 
% \[
% Q = 
% \begin{pmatrix}
% Q_{11} & Q_{12} & \cdots & Q_{1n} \\
% 0 & Q_{22} &\cdots & Q_{2n} \\
% \vdots & \cdot & \ddots & \vdots \\
% 0 & 0 & \cdots & Q_{nn} \\
% \end{pmatrix}
% \]
% \[
% Q^\top Q = I
% \]
% \[
% \begin{pmatrix}
% Q_{11} & 0 & \cdots & 0 \\
% Q_{12} & Q_{22} &\cdots & 0 \\
% \vdots & \cdot & \ddots & \vdots \\
% Q_{1n} &  0 & \cdots & Q_{nn} \\
% \end{pmatrix}
% \begin{pmatrix}
% Q_{11} & Q_{12} & \cdots & Q_{1n} \\
% 0 & Q_{22} &\cdots & Q_{2n} \\
% \vdots & \cdot & \ddots & \vdots \\
% 0 & 0 & \cdots & Q_{nn} \\
% \end{pmatrix}
% =
% \begin{pmatrix}
% 1 & 0 & \cdots & 0 \\
% 0 & 1 &\cdots & 0 \\
% \vdots & \cdot & \ddots & \vdots \\
% 0 & 0 & \cdots & 1 \\
% \end{pmatrix}
% \]
$Q = 
\begin{pmatrix}
Q_1 & Q_2 & \cdots & Q_n \\
\end{pmatrix}
$ and $Q^\top Q = I$ : 
\[
\begin{pmatrix}
Q_1^\top \\ Q_2^\top \\ \cdots \\ Q_n^\top \\
\end{pmatrix}
\begin{pmatrix}
Q_1 & Q_2 & \cdots & Q_n \\
\end{pmatrix} = I
\]
\[
\Rightarrow
\left\{\begin{matrix}
Q_i^\top Q_i = 1 \qquad \qquad \forall i \in  \overline{1,n}\\[4pt]
Q_i^\top Q_j = 0 \quad \forall i,j \in \overline{1,n} ~ i \neq j
\end{matrix}\right.
\]\\
\[
\Rightarrow Q_1^\top Q_1 = Q_{11}^2 = 1
\quad \Rightarrow Q_{11} = \pm 1
\quad \Rightarrow Q_1^\top Q_j = \pm Q_{1j} = 0
~\forall j \neq 1
\]\\
\[
\Rightarrow Q_2^\top Q_2 = Q_{12}^2 + Q_{22}^2 = 0 + Q_{22}^2 = 1
\quad \Rightarrow Q_{22} = \pm 1
\quad \Rightarrow Q_2^\top Q_j = \pm Q_{2j} = 0
~\forall j \neq 2
\]\\
and so on. Hence $Q$ is diagonal and moreover:
\[
Q = 
\begin{pmatrix}
\pm 1 & 0 & \cdots & 0 \\
0 & \pm 1 & \cdots & 0 \\
\vdots & \vdots & \ddots & \vdots \\
0 & 0 & \cdots & \pm 1
\end{pmatrix}
\]\\
\end{enumerate}
\end{solution}



\begin{problem}[Projection matrices; 3pt]\rm For the vectors $\ba_1=(1,0,1)$, $\ba_2 = (0,1,2)$ and $\bb=(-1,2,1)$
	\begin{enumerate}[(a)]
		\item  find the matrix of the orthogonal projection $P_{W}$ onto the plane $W:=\operatorname{ls}\{\ba_1,\ba_2\}$;
		\item  find the matrix of the orthogonal projection $P_{W^\perp}$ onto the line~$W^\perp$;
		\item find the components of the vector~$\bb$ with respect to the decomposition $\bR^3 = W \oplus W^\perp$.
	\end{enumerate}
\end{problem}


\begin{solution}[]\rm .
\begin{enumerate}[(a)]
\item
% find the matrix of the orthogonal projection $P_{W}$ onto the plane $W:=\operatorname{ls}\{\ba_1,\ba_2\}$;
\[
A = 
\begin{pmatrix}
1 & 0 \\
0 & 1 \\
1 & 2 \\
\end{pmatrix}
\]
\[
P_W = A (A^\top A)^{-1} A^\top
\]
\[
A^\top A =
\begin{pmatrix}
1 & 0 & 1\\
0 & 1 & 2 \\
\end{pmatrix}
\begin{pmatrix}
1 & 0 \\
0 & 1 \\
1 & 2 \\
\end{pmatrix}
=
\begin{pmatrix}
2 & 2 \\
2 & 5 \\
\end{pmatrix}
\]
\[
(A^\top A)^{-1} = \frac{1}{6}
\begin{pmatrix}
5 & -2 \\
-2 & 2 \\
\end{pmatrix}
\]
\[
P_W = \frac{1}{6}
\begin{pmatrix}
1 & 0 \\
0 & 1 \\
1 & 2 \\
\end{pmatrix}
\begin{pmatrix}
5 & -2 \\
-2 & 2 \\
\end{pmatrix}
\begin{pmatrix}
1 & 0 & 1\\
0 & 1 & 2 \\
\end{pmatrix}
= \frac{1}{6}
\begin{pmatrix}
1 & 0 \\
0 & 1 \\
1 & 2 \\
\end{pmatrix}
\begin{pmatrix}
5 & -2 & 1\\
-2 & 2 & 2 \\
\end{pmatrix}
= \frac{1}{6}
\begin{pmatrix}
5 & -2 & 1 \\
-2 & 2 & 2 \\
1 & 2 & 5 \\
\end{pmatrix}
\]
\item 
% find the matrix of the orthogonal projection $P_{W^\perp}$ onto the line~$W^\perp$;
\[
W^\perp = \ls\{\ba\}
\]
where:
\[
\ba = \ba_1\times\ba_2 = 
\begin{pmatrix}
i & j & k \\
1 & 0 & 1\\
0 & 1 & 2 \\
\end{pmatrix} = -i -2j + k = (-1, -2, 1)^\top
\]
\[
P_{W^\top} = \frac{\ba\ba^\top}{\ba^\top\ba} = \frac{1}{6}\begin{pmatrix}
1 & 2 & -1 \\
2 & 4 & -2\\
-1 & -2 & 1 \\
\end{pmatrix} 
\]
\item
% find the components of the vector~$\bb$ with respect to the decomposition $\bR^3 = W \oplus W^\perp$.
\[
\bb = P_W\bb + P_{W^\top}\bb
\] \\
\[
P_W\bb = \frac{1}{6}
\begin{pmatrix}
5 & -2 & 1 \\
-2 & 2 & 2 \\
1 & 2 & 5 \\
\end{pmatrix}
\begin{pmatrix}
-1 \\ 2 \\ 1
\end{pmatrix}
 = \frac{1}{6}
 \begin{pmatrix}
-8 \\ 
8 \\ 
8 \\ 
\end{pmatrix}
 =
 \begin{pmatrix}
-4/3 \\ 
 4/3 \\ 
 4/3 \\ 
\end{pmatrix}
\]\\
\[
P_{W^\top}\bb = \frac{1}{6}
\begin{pmatrix}
1 & 2 & -1 \\
2 & 4 & -2\\
-1 & -2 & 1 \\
\end{pmatrix}
\begin{pmatrix}
-1 \\ 2 \\ 1
\end{pmatrix}
= \frac{1}{6}
\begin{pmatrix}
 2 \\ 
 4 \\ 
-2 \\ 
\end{pmatrix} = 
\begin{pmatrix}
 1/3 \\ 
 2/3 \\ 
-1/3 \\ 
\end{pmatrix}
\]
Indeed:
\[
P_W\bb + P_{W^\top}\bb = 
 \begin{pmatrix}
-4/3 \\ 
 4/3 \\ 
 4/3 \\ 
\end{pmatrix}
+
\begin{pmatrix}
 1/3 \\ 
 2/3 \\ 
-1/3 \\ 
\end{pmatrix}
=
\begin{pmatrix}
 -1 \\ 
 2 \\ 
1 \\ 
\end{pmatrix}
= \bb 
\]\\
\end{enumerate}
\end{solution}



\begin{problem}[Least squares solution; 4pt]\rm
	Is there any value of $s$ for which $x_1=1$ and $x_2=2$ is the least squares solution of the linear
	system below?	Explain your reasoning. \\ [-10pt]
	\begin{align*}
	x_1 - \hphantom{4}x_2 & = 1,\\ 2x_1 + 3x_2 & = 1,\\ 4x_1 + 5x_2 & = s.
	\end{align*}
\vspace*{-10pt}
\end{problem}

\begin{solution}[]\rm 
Let
\[
A = 
\begin{pmatrix}
1 & -1 \\
2 & 3 \\
4 & 5 \\
\end{pmatrix} =
\begin{pmatrix}
\ba_1 & \ba_2 \\
\end{pmatrix} \quad
\bb = 
\begin{pmatrix}
1 \\ 1 \\ s \\
\end{pmatrix}
\]
Let
$
\bx^* = 
\begin{pmatrix}
1 \\ 2 \\
\end{pmatrix}
$ be the least squares solution of the linear system $A\bx = \bb$. Therefore $\be = \bb-A\bx^*$ should be orthogonal to $C(A) \Leftrightarrow \sprod{\be}{\ba_i} = 0 \quad i = {1,2}$
\[
\be = \bb - A\bx^* = 
\begin{pmatrix}
1 \\ 1 \\ s \\
\end{pmatrix}
-
\begin{pmatrix}
1 & -1 \\
2 & 3 \\
4 & 5 \\
\end{pmatrix}
\begin{pmatrix}
1 \\ 2 \\
\end{pmatrix}
=
\begin{pmatrix}
1 \\ 1 \\ s \\
\end{pmatrix}
-
\begin{pmatrix}
-1 \\
8 \\
14 \\
\end{pmatrix}
=
\begin{pmatrix}
2 \\ -7 \\ s-14 \\
\end{pmatrix}
\]
\[
\left\{\begin{matrix}
\sprod{\be}{\ba_1} = 
\sprod{
\begin{pmatrix}
2 \\ -7 \\ s-14 \\
\end{pmatrix}
}{
\begin{pmatrix}
1 \\ 2 \\ 4 \\
\end{pmatrix}
}
= 2 - 14  + 4s - 48 = 4s - 60 = 0 \quad \Rightarrow \quad s = 15 \\
\sprod{\be}{\ba_2} = 
\sprod{
\begin{pmatrix}
2 \\ -7 \\ s-14 \\
\end{pmatrix}
}{
\begin{pmatrix}
-1 \\ 3 \\ 5 \\
\end{pmatrix}
}
= -2 - 21  + 5s - 70 = 5s - 93 = 0 ~ \Rightarrow ~ s = 18.6
\end{matrix}\right.
\]
\[
\Rightarrow s \in \varnothing 
\]
Hence there is no value of s for which $\bx^*$ is the least squares solution of $A\bx=\bb$.\\
\end{solution}


\begin{problem}[Regression; 6pt]\rm
	\begin{enumerate}[(a)]
		\item  Find the least squares straight line fit to the four points $(0,1)$, $(2,0)$, $(3,1)$, and $(3,2)$.
		\item  Find the quadratic polynomial that best fits the four points $(2,0)$, $(3,-10)$, $(5,-48)$, and $(6,-76)$.
		\item  Find the cubic polynomial that best fits the five points $(-1,-14)$, $(0,-5)$, $(1,-4)$, $(2,1)$, and $(3,22)$.
	\end{enumerate}
	{\small{\textsf{Hint: the numbers are chosen so that $A^\top A$ can easily be inverted. If, however, this is not so, ask Python or anybody else for a help.}}}
\end{problem}


\begin{solution}[]\rm .
\begin{enumerate}[(a)]
\item
\[
A = 
\begin{pmatrix}
1 & 0\\
1 & 2\\
1 & 3\\
1 & 3\\
\end{pmatrix} \quad
b = 
\begin{pmatrix}
1\\ 0\\ 1\\ 2\\
\end{pmatrix}
\]
The first column of $A$ corresponds to the y-intercept, and the second -- to the 1-degree term.
\[
A^\top A = 
\begin{pmatrix}
 4 &  8 \\
 8 &  22 \\
\end{pmatrix} \qquad
(A^\top A)^{-1} = \frac{1}{12}
\begin{pmatrix}
 11 &  -4 \\
 -4 &   2 \\
\end{pmatrix} \qquad
(A^\top A)^{-1} A^\top =  \frac{1}{12}
\begin{pmatrix}
11 &  3 & -1 & -1 \\
-4 &  0 &  2 &  2 \\
\end{pmatrix}
\]
\[
x = (A^\top A)^{-1} A^\top b
= \frac{1}{12}
\begin{pmatrix} 
8 \\ 2
\end{pmatrix}
=
\begin{pmatrix} 
2/3 \\ 1/6
\end{pmatrix}
\]
So, the quadratic polynomial $y = \frac{1}{6}x + \frac{2}{3}$ best fits the points.
\item
\[
A = 
\begin{pmatrix}
1 & 2 & 4\\
1 & 3 & 9\\
1 & 5 & 25\\
1 & 6 & 36\\
\end{pmatrix} \quad
b = 
\begin{pmatrix}
0\\ -10\\ -48\\ -76\\
\end{pmatrix}
\]
The first column of $A$ corresponds to the y-intercept, the second -- to the 1-degree term, the third -- to the 2-degree term.
\[
A^\top A = 
\begin{pmatrix}
 4 &   16 &   74 \\
16 &   74 &  376 \\
74 &  376 & 2018 \\
\end{pmatrix} \qquad
(A^\top A)^{-1} = \frac{1}{90}
\begin{pmatrix}
 1989 & -1116 & 135 \\ 
-1116 &   649 & -80 \\ 
  135 &   -80 &  10 \\
\end{pmatrix}
\]
\[
(A^\top A)^{-1} A^\top =  \frac{1}{90}
\begin{pmatrix}
 297 & -144 & -216 &  153 \\
-138 &  111 &  129 & -102 \\
  15 &  -15 &  -15 &   15 \\
\end{pmatrix}
\]
\[
x = (A^\top A)^{-1} A^\top b
= \frac{1}{90}
\begin{pmatrix} 
180 \\ 450 \\ -270
\end{pmatrix}
=
\begin{pmatrix} 
2 \\ 5 \\ -3
\end{pmatrix}
\]
So, the quadratic polynomial $y = 3x^2 + 5x + 2$ best fits the points.
\item
\[
A = 
\begin{pmatrix}
1 & -1 & 1 & -1\\
1 & 0 & 0 & 0\\
1 & 1 & 1 & 1\\
1 & 2 & 4 & 8\\
1 & 3 & 9 & 27\\
\end{pmatrix} \quad
b = 
\begin{pmatrix}
-14\\ -5\\ -4\\ 1\\ 22 \\
\end{pmatrix}
\]
The first column of $A$ corresponds to the y-intercept, the second -- to the 1-degree term, the third -- to the 2-degree term, the fourth -- to the 3-degree term.
\[
A^\top A = 
\begin{pmatrix}
 5 &  5 & 15 &  35 \\
 5 & 15 & 35 &  99 \\
15 & 35 & 99 & 275 \\
35 & 99 &275 & 795 \\
\end{pmatrix} \quad
(A^\top A)^{-1} = \frac{1}{2520}
\begin{pmatrix}
 1944 &  -60 & -1440 &  420 \\
  -60 & 1000 &  -150 &  -70 \\
-1440 & -150 &  1755 & -525 \\
  420 &  -70 &  -525 &  175 \\
\end{pmatrix} \qquad
\]
\[
(A^\top A)^{-1} A^\top =  \frac{1}{2520}
\begin{pmatrix}
 144 &  1944 &  864 & -576 &  144 \\
-1140 & -60 &  720 &  780 & -300 \\
 990 & -1440 & -360 & 1080 & -270 \\
-210 &  420 &    0 & -420 &  210 \\
\end{pmatrix}
\]
\[
x = (A^\top A)^{-1} A^\top b
= \frac{1}{2520}
\begin{pmatrix} 
-12600 \\ 7560 \\ -10080 \\ 5040 \\
\end{pmatrix}
=
\begin{pmatrix} 
-5 \\ 3 \\ -4 \\ 2 \\
\end{pmatrix}
\]
So, the cubic polynomial $y = 2 x^3 - 4x^2 + 3x - 5$ best fits the points.\\
\end{enumerate}
\end{solution}



\begin{problem}[Least square solution; 5pt]
	\rm
	Assume $\bu_1$ and $\bu_2$ are two orthogonal vectors in $\bR^n$ and set $\ba_1 = \bu_1$, $\ba_2 = \bu_1 + \varepsilon \bu_2$ for $\varepsilon>0$. Let also $A$ be the matrix with columns $\ba_1$ and $\ba_2$ and $\bb$ a vector linearly independenet of $\ba_1$ and $\ba_2$. In this problem, we discuss the least square solution to the system $A\bx = \bb$ as $\varepsilon\to0$. 
	\begin{enumerate}[(a)]
		\item Calculate the matrix $A^\top A$, its inverse, and then $\hat\bx = (A^\top A)^{-1}A^\top\bb$ explicitly. Show that $\hat\bx$ explodes as $\varepsilon\to0$.
		\item Calculate the projection $A\hat{\bx}$ of $\bb$ onto $\operatorname{col}(A)$ and check that it does not depend on~$\varepsilon>0$. Explain the result.
	\end{enumerate}
\end{problem}

\begin{solution}[]\rm .
\begin{enumerate}[(a)]
\item
\[
A = 
\begin{pmatrix} 
\ba_1 & \ba_2
\end{pmatrix} =
\begin{pmatrix} 
\bu_1 & \bu_1+\varepsilon\bu_2
\end{pmatrix}
\qquad
A^\top =
\begin{pmatrix} 
\ba_1^\top \\ \ba_2^\top
\end{pmatrix} =
\begin{pmatrix} 
\bu_1^\top \\ \bu_1^\top+\varepsilon\bu_2^\top
\end{pmatrix}
\]
\[
A^\top A = 
\begin{pmatrix} 
\ba_1^\top \\ \ba_2^\top
\end{pmatrix}
\begin{pmatrix} 
\ba_1 & \ba_2
\end{pmatrix}
=
\begin{pmatrix} 
\bu_1^\top \\ \bu_1^\top+\varepsilon\bu_2^\top
\end{pmatrix}
\begin{pmatrix} 
\bu_1 & \bu_1+\varepsilon\bu_2
\end{pmatrix}
=
\begin{pmatrix} 
\bu_1^\top \bu_1 & \bu_1^\top (\bu_1+\varepsilon\bu_2) \\
(\bu_1^\top+\varepsilon\bu_2^\top)\bu_1 & (\bu_1^\top+\varepsilon\bu_2^\top)(\bu_1+\varepsilon\bu_2) \\
\end{pmatrix}
=
\]
\[
=
\begin{pmatrix} 
\bu_1^\top \bu_1 &
\bu_1^\top \bu_1 + \varepsilon \bu_1^\top \bu_2\\
\bu_1^\top \bu_1 + \varepsilon\bu_2^\top \bu_1 &
\bu_1^\top \bu_1 + \varepsilon(\bu_2^\top \bu_1 + \bu_1^\top \bu_2) + \varepsilon^2 \bu_2^\top \bu_2\\
\end{pmatrix}
\]
Since $\sprod{\bu_1}{\bu_2} = 0$:
\[
A^\top A =
\begin{pmatrix} 
\norm{\bu_1}^2 & \norm{\bu_1}^2 \\
\norm{\bu_1}^2 & \norm{\bu_1}^2  + \varepsilon^2 \norm{\bu_2}^2\\
\end{pmatrix}
\]\\
\[
\det{A^\top A} = \norm{\bu_1}^2 (\norm{\bu_1}^2  + \varepsilon^2 \norm{\bu_2}^2) - \norm{\bu_1}^4 = \varepsilon^2 \norm{\bu_1}^2 \norm{\bu_2}^2
\]\\
\[
(A^\top A)^{-1} =
\frac{1}{\varepsilon^2 \norm{\bu_1}^2 \norm{\bu_2}^2}
\begin{pmatrix} 
\norm{\bu_1}^2  + \varepsilon^2 \norm{\bu_2}^2 &  -\norm{\bu_1}^2 \\
-\norm{\bu_1}^2 & \norm{\bu_1}^2 \\
\end{pmatrix}
=
\frac{1}{\varepsilon^2\norm{\bu_2}^2}
\begin{pmatrix} 
1 &  -1 \\
-1 & 1 \\
\end{pmatrix}
+
\frac{1}{\norm{\bu_1}^2}
\begin{pmatrix} 
1 &  0 \\
0 & 0 \\
\end{pmatrix}
\]\\
\[
(A^\top A)^{-1}A^\top 
=
\frac{1}{\varepsilon^2\norm{\bu_2}^2}
\begin{pmatrix} 
1 &  -1 \\
-1 & 1 \\
\end{pmatrix}
\begin{pmatrix} 
\bu_1^\top \\ \bu_1^\top+\varepsilon\bu_2^\top
\end{pmatrix}
+
\frac{1}{\norm{\bu_1}^2}
\begin{pmatrix} 
1 &  0 \\
0 & 0 \\
\end{pmatrix}
\begin{pmatrix} 
\bu_1^\top \\ \bu_1^\top+\varepsilon\bu_2^\top
\end{pmatrix}
=
\]
\[
=
\frac{1}{\varepsilon^2\norm{\bu_2}^2}
\begin{pmatrix} 
\bu_1^\top - \bu_1^\top - \varepsilon\bu_2^\top \\
-\bu_1^\top + \bu_1^\top + \varepsilon\bu_2^\top \\
\end{pmatrix}
+
\frac{1}{\norm{\bu_1}^2}
\begin{pmatrix} 
\bu_1^\top \\ 0
\end{pmatrix}
=
\frac{1}{\varepsilon\norm{\bu_2}^2}
\begin{pmatrix} 
 - \bu_2^\top \\
 \bu_2^\top \\
\end{pmatrix}
+
\frac{1}{\norm{\bu_1}^2}
\begin{pmatrix} 
\bu_1^\top \\ 0
\end{pmatrix}
=
\]
\[
=
\frac{1}{\varepsilon\norm{\bu_2}^2}
\begin{pmatrix} 
 - 1 \\
 1 \\
\end{pmatrix}
\bu_2^\top
+
\frac{1}{\norm{\bu_1}^2}
\begin{pmatrix} 
1 \\ 0
\end{pmatrix}
\bu_1^\top
\]\\
When $\bb$ is orthogonal to $\bu_2$, we have:
\[
\hat \bx =
\frac{1}{\varepsilon\norm{\bu_2}^2}
\begin{pmatrix} 
 - 1 \\ 1 \\
\end{pmatrix}
\bu_2^\top \bb
+
\frac{1}{\norm{\bu_1}^2}
\begin{pmatrix} 
1 \\ 0
\end{pmatrix}
\bu_1^\top \bb 
=
\frac{1}{\norm{\bu_1}^2}
\begin{pmatrix} 
1 \\ 0
\end{pmatrix}
\bu_1^\top \bb 
\]
that doesn't depend on $\varepsilon$. In other cases the component 
$
\frac{1}{\varepsilon\norm{\bu_2}^2}
\begin{pmatrix} 
 - 1 \\ 1 \\
\end{pmatrix}
\bu_2^\top \bb
\to \infty$ as $\varepsilon\to0$ which leads to $\hat\bx$ exploding as $\varepsilon\to0$.
% When $\varepsilon \to 0$, the first component 
% \[
% \hat\bx = (A^\top A)^{-1}A^\top\bb = 
% \frac{1}{\varepsilon^2\norm{\bu_2}^2}
% \begin{pmatrix} 
% 1 &  -1 \\
% -1 & 1 \\
% \end{pmatrix}
% \begin{pmatrix} 
% \bu_1^\top \\ \bu_1^\top+\varepsilon\bu_2^\top
% \end{pmatrix}
% \bb
% +
% \frac{1}{\norm{\bu_1}^2}
% \begin{pmatrix} 
% 1 &  0 \\
% 0 & 0 \\
% \end{pmatrix}
% \begin{pmatrix} 
% \bu_1^\top \\ \bu_1^\top+\varepsilon\bu_2^\top
% \end{pmatrix}
% \bb
% \]\\
\item
The projection matrix:
\[
P
% A\hat\bx 
% =
% \frac{1}{\varepsilon\norm{\bu_2}^2}
% \begin{pmatrix} 
% \bu_1 & \bu_1+\varepsilon\bu_2
% \end{pmatrix}
% \begin{pmatrix} 
%  - \bu_2^\top \\
%  \bu_2^\top \\
% \end{pmatrix}
% +
% \frac{1}{\norm{\bu_1}^2}
% \begin{pmatrix} 
% \bu_1 & \bu_1+\varepsilon\bu_2
% \end{pmatrix}
% \begin{pmatrix} 
% \bu_1^\top \\ 0
% \end{pmatrix}
=
\frac{1}{\varepsilon\norm{\bu_2}^2}
\begin{pmatrix} 
\bu_1 & \bu_1+\varepsilon\bu_2
\end{pmatrix}
\begin{pmatrix} 
 - 1 \\
 1 \\
\end{pmatrix}
\bu_2^\top
+
\frac{1}{\norm{\bu_1}^2}
\begin{pmatrix} 
\bu_1 & \bu_1+\varepsilon\bu_2
\end{pmatrix}
\begin{pmatrix} 
1 \\ 0
\end{pmatrix}
\bu_1^\top
=
\]
\[
=
\frac{1}{\varepsilon\norm{\bu_2}^2}
\varepsilon\bu_2
\bu_2^\top
+
\frac{1}{\norm{\bu_1}^2}
\bu_1
\bu_1^\top
=
\frac{\bu_2\bu_2^\top}{\norm{\bu_2}^2}
+
\frac{\bu_1\bu_1^\top}{\norm{\bu_1}^2}
\]
does not depend on $\varepsilon$ so neither does the projection $P\bb$.\\
\end{enumerate}
\end{solution}


\begin{problem}[Gram--Schmidt; 3pt]\rm
	Use the Gram--Schmidt process to transform the basis $\bu_1,\dots,\bu_k$ into
	an orthonormal basis.
	\begin{enumerate}[(a)]
		\item $\bu_1=(1,3)$, $\bu_2=(2,-2)$;
		\item $\bu_1=(1,0,1)$, $\bu_2 = (1,3,-2)$, $\bu_3=(0,2,1)$
	\end{enumerate}
\end{problem}

\begin{solution}[]\rm .
\begin{enumerate}[(a)]
\item $\bu_1=(1,3)$, $\bu_2=(2,-2)$
\[
\bw_1 = \bu_1 =
\begin{pmatrix} 
1 \\ 3
\end{pmatrix}
\qquad
\bw_2 = \bu_2 - \frac{\bw_1\bw_1^\top}{\bw_1^\top\bw_1}\bu_2
\]
\[
\frac{\bw_1\bw_1^\top}{\bw_1^\top\bw_1} =
\frac{1}{10} 
\begin{pmatrix} 
1 \\ 3
\end{pmatrix}
\begin{pmatrix} 
1 & 3
\end{pmatrix}
=
\frac{1}{10} 
\begin{pmatrix} 
1 & 3 \\
3 & 9 \\
\end{pmatrix}
\]
\[
\bw_2 =
\begin{pmatrix} 
2 \\ -2
\end{pmatrix}
-
\frac{1}{10} 
\begin{pmatrix} 
1 & 3 \\
3 & 9 \\
\end{pmatrix}
\begin{pmatrix} 
2 \\ -2
\end{pmatrix} =
\frac{1}{10} 
\begin{pmatrix}
20 -2 + 6 \\-20  - 6 + 18 
\end{pmatrix} =
\frac{1}{5} 
\begin{pmatrix}
12 \\ -4
\end{pmatrix} 
\]
\[
\hat \bw_1 = \frac{\bw_1}{\norm \bw_1} = 
\frac{1}{\sqrt 10} 
\begin{pmatrix} 
1 \\ 3
\end{pmatrix}
\]
\[
\hat \bw_2 = \frac{\bw_2}{\norm \bw_2} = 
\frac{1}{\sqrt {10}}
\begin{pmatrix} 
3 \\ -1
\end{pmatrix}
\]
So the answer is $\hat \bw_1 =
\begin{pmatrix}  1/\sqrt 10 \\ 3/\sqrt 10\end{pmatrix}$,
$\hat \bw_2 = 
\begin{pmatrix} 3/\sqrt 10 \\ -1/\sqrt 10 \end{pmatrix}$
\item $\bu_1=(1,0,1)$, $\bu_2 = (1,3,-2)$, $\bu_3=(0,2,1)$
\[
\bw_1 = \bu_1 = \begin{pmatrix} 1 \\ 0 \\ 1 \end{pmatrix}
\qquad
\bw_2 = \bu_2 - \frac{\bw_1\bw_1^\top}{\bw_1^\top\bw_1}\bu_2
\qquad
\bw_3 = \bu_3 - \frac{\bw_1\bw_1^\top}{\bw_1^\top\bw_1}\bu_3 - \frac{\bw_2\bw_2^\top}{\bw_2^\top\bw_2}\bu_3
\]
\[
\frac{\bw_1\bw_1^\top}{\bw_1^\top\bw_1} =
\frac{1}{2}
\begin{pmatrix} 
 1 \\ 0 \\ 1 
\end{pmatrix}
\begin{pmatrix}
 1 & 0 & 1
\end{pmatrix}
=
\frac{1}{2}
\begin{pmatrix} 
 1 & 0 & 1 \\
 0 & 0 & 0 \\
 1 & 0 & 1 \\
\end{pmatrix}
\]
\[
\bw_2 = 
\begin{pmatrix} 
 1 \\ 3 \\ -2
\end{pmatrix}
 - 
\frac{1}{2}
\begin{pmatrix} 
 1 & 0 & 1 \\
 0 & 0 & 0 \\
 1 & 0 & 1 \\
\end{pmatrix}
\begin{pmatrix} 
 1 \\ 3 \\ -2
\end{pmatrix}
=
\frac{1}{2}
\begin{pmatrix} 
 2 - 1 + 2\\ 6 - 0 \\ -4 - 1 + 2
\end{pmatrix}
=
\frac{1}{2}
\begin{pmatrix} 
 3 \\ 6 \\ -3
\end{pmatrix}
\]
\[
\frac{\bw_2\bw_2^\top}{\bw_2^\top\bw_2} =
\frac{1}{54}
\begin{pmatrix} 
 3 \\ 6 \\ -3
\end{pmatrix}
\begin{pmatrix}
 3 & 6 & -3
\end{pmatrix}
=
\frac{1}{54}
\begin{pmatrix} 
 9 & 18 & -9 \\
 18 & 36 & -18 \\
 -9 & -18 & 9 \\
\end{pmatrix}
=
\frac{1}{6}
\begin{pmatrix} 
 1 & 2 & -1 \\
 2 & 4 & -2 \\
 -1 & -2 & 1 \\
\end{pmatrix}
\]
\[
\bw_3 = 
\begin{pmatrix} 
 0 \\ 2 \\ 1
\end{pmatrix}
 - 
\frac{1}{2}
\begin{pmatrix} 
 1 & 0 & 1 \\
 0 & 0 & 0 \\
 1 & 0 & 1 \\
\end{pmatrix}
\begin{pmatrix} 
 0 \\ 2 \\ 1
\end{pmatrix}
-
\frac{1}{6}
\begin{pmatrix} 
 1 & 2 & -1 \\
 2 & 4 & -2 \\
 -1 & -2 & 1 \\
\end{pmatrix}
\begin{pmatrix} 
 0 \\ 2 \\ 1
\end{pmatrix}
=
\frac{1}{6}
\begin{pmatrix} 
 0-3- 4 + 1 \\ 12-0- 8 + 2 \\ 6-3+ 4 - 1
\end{pmatrix}
=
\begin{pmatrix} 
 -1 \\ 1 \\ 1
\end{pmatrix}
\]
\[
\hat \bw_1 = \frac{\bw_1}{\norm \bw_1} = 
\frac{1}{\sqrt 2}
\begin{pmatrix}
1 \\ 0 \\ 1
\end{pmatrix}
\]
\[
\hat \bw_2 = \frac{\bw_2}{\norm \bw_2} = 
\frac{1}{3\sqrt 6}
\begin{pmatrix} 
 3 \\ 6 \\ -3
\end{pmatrix}
\]
\[
\hat \bw_3 = \frac{\bw_3}{\norm \bw_3} = 
\frac{1}{\sqrt 3}
\begin{pmatrix} 
 -1 \\ 1 \\ 1
\end{pmatrix}
\]
So the answer is $\hat \bw_1 =
\begin{pmatrix} 1/\sqrt 2 \\ 0 \\ 1/\sqrt 2 \end{pmatrix}$, 
$\hat \bw_2 = 
\begin{pmatrix}  1/\sqrt 6 \\ 2/\sqrt 6 \\ -1/\sqrt 6 \end{pmatrix}$,
$\hat \bw_3 = 
\begin{pmatrix}  -1/\sqrt 3 \\ 1/\sqrt 3 \\ 1/\sqrt 3 \end{pmatrix}$.\\
\end{enumerate}
\end{solution}


\begin{problem}[QR; 5pt]\label{prb:QR}\rm
	Find the $QR$-decomposition of the matrices below using the Gram--Schmidt algorithm:
	\[
	(a)~\begin{pmatrix}[rr] 1 & -1 \\ 2 & 3  \end{pmatrix}; 
	\qquad
	(b)~\begin{pmatrix}[rr] 1 & 2 \\ 0 & 1 \\ 1 & 4 \end{pmatrix}; 
	\qquad
	(c)~\begin{pmatrix}[rrr] 1 & 0 & 2 \\ 0 & 1 & 1 \\ 2 & 0 & 1 \end{pmatrix}
	\]
\end{problem}	

\begin{solution}[]\rm .
\begin{enumerate}[(a)]
\item
\[
A = 
\begin{pmatrix}
1 & -1 \\
2 & 3
\end{pmatrix}
\]
\[
q_1 = \begin{pmatrix} 1 \\ 2  \end{pmatrix}
\qquad
\frac{q_1 q_1^\top}{q_1^\top q_1} =
\frac{1}{5}
\begin{pmatrix} 1 & 2 \\ 2 & 4 \end{pmatrix}
\]
\[
q_2 = \begin{pmatrix} -1 \\ 3  \end{pmatrix} - 
\frac{1}{5}
\begin{pmatrix} 1 & 2 \\ 2 & 4 \end{pmatrix}
\begin{pmatrix} -1 \\ 3  \end{pmatrix}
= \frac{1}{5}
\begin{pmatrix} -5 + 1 - 6 \\ 15 + 2 - 12 \end{pmatrix}
=
\begin{pmatrix} -2 \\ 1 \end{pmatrix}
\]
\[
\hat q_1 = \begin{pmatrix} 1/\sqrt5 \\ 2/\sqrt5  \end{pmatrix}
\qquad
\hat q_2 = \begin{pmatrix} -2/\sqrt5 \\ 1/\sqrt5 \end{pmatrix}
% \]
% \[
\qquad
Q =
\begin{pmatrix}
1/\sqrt5 & -2/\sqrt5 \\
2/\sqrt5 & 1/\sqrt5
\end{pmatrix}
\]
\[
R =
\begin{pmatrix}
\hat q_1^\top a_1 & \hat q_1^\top a_2 \\
0 & \hat q_2^\top a_2 \\
\end{pmatrix}
=
\begin{pmatrix}
5/\sqrt 5 & 5/\sqrt 5 \\
0 & 5/\sqrt 5 \\
\end{pmatrix}
\]
\[
\Rightarrow A = 
\begin{pmatrix}
1 & -1 \\
2 & 3
\end{pmatrix}
=
\begin{pmatrix}
1/\sqrt5 & -2/\sqrt5 \\
2/\sqrt5 & 1/\sqrt5
\end{pmatrix}
\begin{pmatrix}
\sqrt 5 & \sqrt 5 \\
0 & \sqrt 5 \\
\end{pmatrix}
\]
\item
\[
A =
\begin{pmatrix}
1 & 2 \\
0 & 1 \\
1 & 4 \end{pmatrix}
\]
\[
q_1 = \begin{pmatrix} 1 \\ 0 \\ 1  \end{pmatrix}
\qquad
\frac{q_1 q_1^\top}{q_1^\top q_1} =
\frac{1}{2}
\begin{pmatrix} 1 & 0 & 1 \\ 0 & 0 & 0 \\ 1 & 0 & 1 \end{pmatrix}
\]
\[
q_2 = \begin{pmatrix} 2 \\ 1 \\ 4  \end{pmatrix} -
\frac{1}{2}
\begin{pmatrix} 1 & 0 & 1 \\ 0 & 0 & 0 \\ 1 & 0 & 1 \end{pmatrix}
\begin{pmatrix} 2 \\ 1 \\ 4  \end{pmatrix}
=
\frac{1}{2}
\begin{pmatrix} 4 - 2 - 4 \\ 2 - 0 \\ 8 - 2 - 4 \end{pmatrix}
=
\begin{pmatrix} - 1 \\ 1 \\ 1 \end{pmatrix}
\]
\[
\hat q_1 = \begin{pmatrix} 1/\sqrt 2 \\ 0 \\ 1/\sqrt 2  \end{pmatrix}
\qquad
\hat q_2 = \begin{pmatrix} - 1/\sqrt 3 \\ 1/\sqrt 3 \\ 1/\sqrt 3 \end{pmatrix}
% \]
% \[
\qquad
Q =
\begin{pmatrix}
1/\sqrt 2 &  - 1/\sqrt 3 \\
0  & 1/\sqrt 3 \\
1/\sqrt 2 & 1/\sqrt 3
\end{pmatrix}
\]
\[
R =
\begin{pmatrix}
\hat q_1^\top a_1 & \hat q_1^\top a_2 \\
0 & \hat q_2^\top a_2 \\
\end{pmatrix}
=
\begin{pmatrix}
2/\sqrt 2 & 6/\sqrt 2 \\
0 & 3/\sqrt 3 \\
\end{pmatrix}
\]
\[
\Rightarrow A = 
\begin{pmatrix}
1 & 2 \\
0 & 1 \\
1 & 4 \end{pmatrix}
=
\begin{pmatrix}
1/\sqrt 2 &  - 1/\sqrt 3 \\
0  & 1/\sqrt 3 \\
1/\sqrt 2 & 1/\sqrt 3
\end{pmatrix}
\begin{pmatrix}
\sqrt 2 & 3\sqrt 2 \\
0 & \sqrt 3 \\
\end{pmatrix}
\]
\item
\[
A =
\begin{pmatrix}
1 & 0 & 2 \\
0 & 1 & 1 \\
2 & 0 & 1
\end{pmatrix}
\]
\[
q_1 = \begin{pmatrix} 1 \\ 0 \\ 2  \end{pmatrix}
\qquad
\frac{q_1 q_1^\top}{q_1^\top q_1} =
\frac{1}{5}
\begin{pmatrix} 1 & 0 & 2 \\ 0 & 0 & 0 \\ 2 & 0 & 4 \end{pmatrix}
\]
\[
q_2 = \begin{pmatrix} 0 \\ 1 \\ 0  \end{pmatrix} -
\frac{1}{5}
\begin{pmatrix} 1 & 0 & 2 \\ 0 & 0 & 0 \\ 2 & 0 & 4 \end{pmatrix}
\begin{pmatrix} 0 \\ 1 \\ 0  \end{pmatrix}
=
\begin{pmatrix} 0 \\ 1 \\ 0  \end{pmatrix}
\qquad
\frac{q_2 q_2^\top}{q_2^\top q_2} =
\begin{pmatrix} 0 & 0 & 0 \\ 0 & 1 & 0 \\ 0 & 0 & 0 \end{pmatrix}
\]
\[
q_3 = \begin{pmatrix} 2 \\ 1 \\ 1  \end{pmatrix} -
-
\frac{1}{5}
\begin{pmatrix} 1 & 0 & 2 \\ 0 & 0 & 0 \\ 2 & 0 & 4 \end{pmatrix}
\begin{pmatrix} 2 \\ 1 \\ 1  \end{pmatrix}
-
\begin{pmatrix} 0 & 0 & 0 \\ 0 & 1 & 0 \\ 0 & 0 & 0 \end{pmatrix}
\begin{pmatrix} 2 \\ 1 \\ 1  \end{pmatrix}
=
\frac{1}{5}
\begin{pmatrix} 10 - 2 - 2 \\ 5 - 5 - 0 \\ 5 - 4 - 4 \end{pmatrix}
=
\frac{1}{5}
\begin{pmatrix} 6 \\ 0 \\ -3 \end{pmatrix}
\]
\[
\hat q_1 =
\begin{pmatrix} 1/\sqrt 5 \\ 0 \\ 2/\sqrt 5  \end{pmatrix}
\qquad
\hat q_2 =
\begin{pmatrix} 0 \\ 1 \\ 0  \end{pmatrix}
\qquad
\hat q_3 =
\begin{pmatrix} 2/\sqrt 5 \\ 0 \\ -1/\sqrt 5  \end{pmatrix}
% \]
% \[
\qquad
Q = 
\begin{pmatrix}
1/\sqrt 5 & 0 & 2/\sqrt 5\\
0 & 1 & 0 \\
2/\sqrt 5 & 0 & -1/\sqrt 5
\end{pmatrix}
\]
\[
R =
\begin{pmatrix}
\hat q_1^\top a_1 & \hat q_1^\top a_2 & \hat q_1^\top a_3 \\
0 & \hat q_2^\top a_2 & \hat q_2^\top a_3\\
0 & \hat 0 & \hat q_3^\top a_3\\
\end{pmatrix}
=
\begin{pmatrix}
5/\sqrt 5 & 0 & 4/\sqrt 5\\
0 & 1 & 1\\
0 & 0 & 3/\sqrt 5\\
\end{pmatrix}
\]
\[
\Rightarrow A = 
\begin{pmatrix}
1 & 0 & 2 \\
0 & 1 & 1 \\
2 & 0 & 1
\end{pmatrix}
=
\begin{pmatrix}
1/\sqrt 5 & 0 & 2/\sqrt 5\\
0 & 1 & 0 \\
2/\sqrt 5 & 0 & -1/\sqrt 5
\end{pmatrix}
\begin{pmatrix}
\sqrt 5 & 0 & 4/\sqrt 5\\
0 & 1 & 1\\
0 & 0 & 3/\sqrt 5\\
\end{pmatrix}\\
\]
\end{enumerate}
\end{solution}


\begin{problem}[Householder reflection and QR; 7 pts]\rm
	\begin{enumerate}[(a)]
		\item Find the unit vector $\bu\in \bR^2$ such that the \emph{Householder reflection} $Q_{\bu}:=I - 2\bu\bu^\top$ maps the vector~$(1,2)^\top$ onto a vector collinear to $(1,0)^\top$
		\item explain how $Q_{\bu}$ helps to derive the $QR$ factorization of the matrix (a) of Problem~\ref{prb:QR}.
		\item Find the $QR$-factorization of matrices in (b) and (c) of Problem~\ref{prb:QR} using the Householder reflections approach.
	\end{enumerate}
\end{problem}


\begin{solution}[]\rm .
\begin{enumerate}[(a)]
\item
% Find the unit vector $\bu\in \bR^2$ such that the \emph{Householder reflection} $Q_{\bu}:=I - 2\bu\bu^\top$ maps the vector~$(1,2)^\top$ onto a vector collinear to $(1,0)^\top$
\[
\bx =  \begin{pmatrix} 1 \\ 2 \end{pmatrix}
\quad 
Q_{\bu}\bx = \norm{\bx} \begin{pmatrix} 1 \\ 0 \end{pmatrix}
\quad
Q_{\bu} = I -  2\bu\bu^\top
\quad
(\norm{\bu} = 1)
\]
To get $\bu$ we need to subtract $\begin{pmatrix} 1 \\ 0 \end{pmatrix}$ from $\bx$ and normalize the vector:
\[
\hat \bu =
\begin{pmatrix} 1 \\ 2 \end{pmatrix}
-
\sqrt 5 \begin{pmatrix} 1 \\ 0 \end{pmatrix}
=
\begin{pmatrix} 1 - \sqrt 5 \\ 2 \end{pmatrix}
\qquad 
\bu = \frac{\hat \bu}{\norm{\hat \bu}} =
\frac{1}{\sqrt{10 - 2 \sqrt 5}}
\begin{pmatrix} 1 - \sqrt 5 \\ 2 \end{pmatrix}
\]
\[
\bu\bu^\top =
\frac{1}{10 - 2 \sqrt 5}
\begin{pmatrix} 1 - \sqrt 5 \\ 2 \end{pmatrix}
\begin{pmatrix} 1 - \sqrt 5 & 2 \end{pmatrix}
=
\frac{1}{10 - 2 \sqrt 5}
\begin{pmatrix}
6 - 2\sqrt 5 & 2 - 2\sqrt 5 \\
2 - 2\sqrt 5  & 4\\
\end{pmatrix} =
\]
\[
=
\frac{1}{5 - \sqrt 5}
\begin{pmatrix}
3 - \sqrt 5 & 1 - \sqrt 5 \\
1 - \sqrt 5  & 2\\
\end{pmatrix}
=\frac{5 + \sqrt 5}{20}
\begin{pmatrix}
3 - \sqrt 5 & 1 - \sqrt 5 \\
1 - \sqrt 5  & 2\\
\end{pmatrix}
=\frac{1}{10}
\begin{pmatrix}
5 - \sqrt 5 & - 2\sqrt 5 \\
- 2\sqrt 5  & 5 + \sqrt 5\\
\end{pmatrix}
\]
\[
\Rightarrow~
Q_{\bu} =
\begin{pmatrix}
1 & 0 \\
0 & 1 \\
\end{pmatrix}
-
\frac{1}{5}
\begin{pmatrix}
5 - \sqrt 5 & - 2\sqrt 5 \\
- 2\sqrt 5  & 5 + \sqrt 5\\
\end{pmatrix}
=
\frac{1}{5}
\begin{pmatrix}
\sqrt 5 & 2 \sqrt 5 \\
2\sqrt 5  & - \sqrt 5\\
\end{pmatrix}
=
\frac{\sqrt 5}{5}
\begin{pmatrix}
1 & 2 \\
2  & - 1\\
\end{pmatrix}
\]
And we can check:
\[
Q_{\bu} \bx = 
\frac{\sqrt 5}{5}
\begin{pmatrix}
1 & 2 \\
2  & - 1\\
\end{pmatrix}
\begin{pmatrix} 1 \\ 2 \end{pmatrix}
= 
\sqrt 5 \begin{pmatrix} 1 \\ 0\\ \end{pmatrix}
~ \parallel ~
\begin{pmatrix} 1 \\ 0 \end{pmatrix}
\qquad 
\norm{Q_{\bu} \bx } = \sqrt 5 = \norm{\bx }
\]
So the answer is:
\[
\bu = \frac{1}{\sqrt{10 - 2 \sqrt 5}}
\begin{pmatrix} 1 - \sqrt 5 \\ 2 \end{pmatrix}
\]
\item
\[
A = 
\begin{pmatrix}
1 & -1 \\
2 & 3
\end{pmatrix}
\]
Since $\bx = \ba_1$, $Q_\bu$ is thus the result of the first iteration of QR factorization using the Householder reflections and:
\[
Q_{\bu}A =
\begin{pmatrix} \sqrt 5 & \sqrt 5 \\ 0 & - \sqrt 5 \end{pmatrix}
\]
give us the first row $\begin{pmatrix} \sqrt 5 & \sqrt 5 \end{pmatrix}$ of the $R$ matrix . 
\item $QR$-factorization of Problem~\ref{prb:QR} (b), (c) using the Householder reflections approach: \\
11. (b)
\[
A =
\begin{pmatrix}
1 & 2 \\
0 & 1 \\
1 & 4 \end{pmatrix}
\qquad
\ba_1 = \begin{pmatrix} 1 \\ 0  \\ 1  \end{pmatrix}
\]
\[
\hat \bu_1 = \ba_1 - \norm{\ba_1} \be_1 = 
\begin{pmatrix} 1 \\ 0  \\ 1  \end{pmatrix}
-
\sqrt 2
\begin{pmatrix} 1 \\ 0  \\ 0  \end{pmatrix}
=
\begin{pmatrix} 1 - \sqrt 2 \\ 0  \\ 1  \end{pmatrix}
\qquad
\bu_1 =
\frac{1}{\sqrt{4 - 2\sqrt 2}}
\begin{pmatrix} 1 - \sqrt 2 \\ 0  \\ 1  \end{pmatrix}
\]
\[
\bu_1\bu_1^\top = 
\frac{1}{4 - 2\sqrt 2}
\begin{pmatrix} 1 - \sqrt 2 \\ 0  \\ 1  \end{pmatrix}
\begin{pmatrix} 1 - \sqrt 2 & 0  & 1  \end{pmatrix}
=
\frac{1}{4 - 2\sqrt 2}
\begin{pmatrix}
3 - 2\sqrt 2 & 0 & 1 - \sqrt 2\\
0 & 0 & 0 \\
1 - \sqrt 2 & 0  & 1 
\end{pmatrix} =
\]\[
=
\frac{4 + 2\sqrt 2}{8}
\begin{pmatrix}
3 - 2\sqrt 2 & 0 & 1 - \sqrt 2\\
0 & 0 & 0 \\
1 - \sqrt 2 & 0  & 1 
\end{pmatrix}
=
\frac{1}{4}
\begin{pmatrix}
2 - \sqrt 2 & 0 & - \sqrt 2\\
0 & 0 & 0 \\
- \sqrt 2 & 0  & 2 + \sqrt 2
\end{pmatrix}
\]
\[
\Rightarrow ~
Q_{\bu_1} =
\frac{1}{2}
\begin{pmatrix}
\sqrt 2 & 0 & \sqrt 2\\
0 & 2 & 0 \\
\sqrt 2 & 0  & -\sqrt 2
\end{pmatrix}
\]
\[
Q_{\bu_1} A = 
\frac{1}{2}
\begin{pmatrix}
\sqrt 2 & 0 & \sqrt 2\\
0 & 2 & 0 \\
\sqrt 2 & 0  & -\sqrt 2
\end{pmatrix}
\begin{pmatrix}
1 & 2 \\
0 & 1 \\
1 & 4 \end{pmatrix}
=
\begin{pmatrix}
\sqrt 2 & 3\sqrt 2\\
0 & 1 \\
0 & -\sqrt 2
\end{pmatrix}
\]\\
\[
\ba_2 = \begin{pmatrix} 1 \\ -\sqrt 2  \end{pmatrix} (\in \mathbb{R}^2)
\]
\[
\hat \bu_2 = \ba_2 - \norm{\ba_2} \be_1 = 
\begin{pmatrix} 1 \\ -\sqrt 2  \end{pmatrix}
-
\sqrt 3
\begin{pmatrix} 1 \\ 0  \end{pmatrix}
=
\begin{pmatrix} 1 - \sqrt 3 \\ - \sqrt 2  \end{pmatrix}
\qquad
\bu_2 =
\frac{1}{\sqrt{6 - 2\sqrt 3}}
\begin{pmatrix} 1 - \sqrt 3 \\ - \sqrt 2  \end{pmatrix}
\]
\[
\bu_2\bu_2^\top = 
\frac{1}{6 - 2\sqrt 3}
\begin{pmatrix} 1 - \sqrt 3 \\ - \sqrt 2  \end{pmatrix}
\begin{pmatrix} 1 - \sqrt 3 & - \sqrt 2  \end{pmatrix}
=
\frac{1}{6 - 2\sqrt 3}
\begin{pmatrix}
4 - 2\sqrt 3 & - \sqrt 2 (1 - \sqrt 3) \\
 - \sqrt 2 (1 - \sqrt 3) & 2\\
\end{pmatrix}
 =
\]\[
=
\frac{3 + \sqrt 3}{12}
\begin{pmatrix}
4 - 2\sqrt 3 & - \sqrt 2 (1 - \sqrt 3) \\
 - \sqrt 2 (1 - \sqrt 3) & 2\\
\end{pmatrix}
 =
\frac{1}{6}
\begin{pmatrix}
3 - \sqrt 3 & \sqrt 6 \\
 \sqrt 6  & 3 + \sqrt 3\\
\end{pmatrix}
\]
\[
\Rightarrow ~
\hat Q_{\bu_2} =
\frac{1}{3}
\begin{pmatrix}
\sqrt 3 & - \sqrt 6 \\
 -\sqrt 6 & - \sqrt 3\\
\end{pmatrix}
~\Rightarrow ~
Q_{\bu_2} =
\frac{1}{3}
\begin{pmatrix}
3 & 0 & 0 \\
0 & \sqrt 3 & - \sqrt 6 \\
0 &  -\sqrt 6 & - \sqrt 3\\
\end{pmatrix}
\]
\[
Q_{\bu_2} Q_{\bu_1} A = 
\frac{1}{3}
\begin{pmatrix}
3 & 0 & 0 \\
0 & \sqrt 3 & - \sqrt 6 \\
0 &  -\sqrt 6 & - \sqrt 3\\
\end{pmatrix}
\begin{pmatrix}
\sqrt 2 & 3\sqrt 2\\
0 & 1 \\
0 & -\sqrt 2
\end{pmatrix}
=
\begin{pmatrix}
\sqrt 2 & 3\sqrt 2 \\
0 & \sqrt 3 \\
0 & 0 \\
\end{pmatrix}
\]
So the full QR factorization is:
\[
R = 
\begin{pmatrix}
\sqrt 2 & 3\sqrt 2 \\
0 & \sqrt 3 \\
0 & 0 \\
\end{pmatrix}
\qquad
Q = Q_{\bu_1} Q_{\bu_2} = 
\frac{1}{6}
\begin{pmatrix}
\sqrt 2 & 0 & \sqrt 2\\
0 & 2 & 0 \\
\sqrt 2 & 0  & -\sqrt 2
\end{pmatrix}
\begin{pmatrix}
3 & 0 & 0 \\
0 & \sqrt 3 & - \sqrt 6 \\
0 &  -\sqrt 6 & - \sqrt 3\\
\end{pmatrix}
=
\]
\[
=
\begin{pmatrix} 
1/\sqrt 2 & -1/\sqrt 3 & - 1/ \sqrt 6 \\
0 & 1/\sqrt 3 & - 2/ \sqrt 6 \\
1/\sqrt 2 & 1/\sqrt 3  & 1/\sqrt 6 \\
\end{pmatrix}
\]
And the reduced QR factorization is:
\[
Q =
\begin{pmatrix} 
1/\sqrt 2 & -1/\sqrt 3\\
0 & 1/\sqrt 3 \\
1/\sqrt 2 & 1/\sqrt 3 \\
\end{pmatrix}
\qquad
R = 
\begin{pmatrix}
\sqrt 2 & 3\sqrt 2 \\
0 & \sqrt 3 \\
\end{pmatrix}
\]
that matches the result of GS approach.\\
11. (c)
\[
A =
\begin{pmatrix}
1 & 0 & 2 \\
0 & 1 & 1 \\
2 & 0 & 1
\end{pmatrix}
\quad
\ba_1 = \begin{pmatrix} 1 \\ 0  \\ 2  \end{pmatrix}
\]
\[
\hat \bu_1 = \ba_1 - \norm{\ba_1} \be_1 = 
\begin{pmatrix} 1 \\ 0  \\ 2  \end{pmatrix}
-
\sqrt 5
\begin{pmatrix} 1 \\ 0  \\ 0  \end{pmatrix}
=
\begin{pmatrix} 1 - \sqrt 5 \\ 0  \\ 2  \end{pmatrix}
\qquad
\bu_1 =
\frac{1}{\sqrt{10 - 2\sqrt 5}}
\begin{pmatrix} 1 - \sqrt 5 \\ 0  \\ 2  \end{pmatrix}
\]
\[
\bu_1\bu_1^\top = 
\frac{1}{10 - 2\sqrt 5}
\begin{pmatrix} 1 - \sqrt 5 \\ 0  \\ 2  \end{pmatrix}
\begin{pmatrix} 1 - \sqrt 5 & 0  & 2  \end{pmatrix}
=
\frac{1}{5 - \sqrt 5}
\begin{pmatrix}
3 - \sqrt 5 & 0 & 1 - \sqrt 5\\
0 & 0 & 0 \\
1 - \sqrt 5 & 0 & 2
\end{pmatrix} =
\]\[
=
\frac{5 + \sqrt 5}{20}
\begin{pmatrix}
3 - \sqrt 5 & 0 & 1 - \sqrt 5\\
0 & 0 & 0 \\
1 - \sqrt 5 & 0 & 2
\end{pmatrix} =
\frac{1}{10}
\begin{pmatrix}
5 - \sqrt 5 & 0 & - 2\sqrt 5\\
0 & 0 & 0 \\
- 2\sqrt 5 & 0 & 5 + \sqrt 5
\end{pmatrix}
\]
\[
\Rightarrow ~
Q_{\bu_1} =
\frac{1}{5}
\begin{pmatrix}
\sqrt 5 & 0 & 2\sqrt 5\\
0 & 5 & 0 \\
2\sqrt 5 & 0 & - \sqrt 5
\end{pmatrix}
\]
\[
Q_{\bu_1}A =
\frac{1}{5}
\begin{pmatrix}
\sqrt 5 & 0 & 2\sqrt 5\\
0 & 5 & 0 \\
2\sqrt 5 & 0 & - \sqrt 5
\end{pmatrix}
\begin{pmatrix}
1 & 0 & 2 \\
0 & 1 & 1 \\
2 & 0 & 1
\end{pmatrix}
=
\begin{pmatrix}
\sqrt 5 & 0 & \frac{4}{5}\sqrt 5\\
0 & 1 & 1 \\
0 & 0 & \frac{3}{5}\sqrt 5
\end{pmatrix}
\]
\[
\ba_2 = \begin{pmatrix} 1 \\ 0  \end{pmatrix} = \be_1 (\in \mathbb{R}^2)
~\Rightarrow~
Q_{\bu_2} = I
\]
\[
Q_{\bu_2}Q_{\bu_1}A =
\begin{pmatrix}
\sqrt 5 & 0 & 4/\sqrt 5\\
0 & 1 & 1 \\
0 & 0 & 3/\sqrt 5
\end{pmatrix}
\]
\[
~\Rightarrow~
R =
\begin{pmatrix}
\sqrt 5 & 0 & 4/\sqrt 5\\
0 & 1 & 1 \\
0 & 0 & 3/\sqrt 5
\end{pmatrix}
\qquad
Q = Q_{\bu_1} Q_{\bu_2} = Q_{\bu_1} = 
\begin{pmatrix}
1/\sqrt 5 & 0 & 2/\sqrt 5\\
0 & 1 & 0 \\
2/\sqrt 5 & 0 & - 1/\sqrt 5
\end{pmatrix}
\]
that matches the result of GS approach.\\
\end{enumerate}
\end{solution}

\end{document}

